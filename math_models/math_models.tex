\documentclass[a4paper,12pt]{extreport}

\usepackage{extsizes}
\usepackage{cmap} % для кодировки шрифтов в pdf
\usepackage[T2A]{fontenc}
\usepackage[utf8]{inputenc}
\usepackage[english]{babel}

\usepackage[usenames, dvipsnames]{color}
\definecolor{fontColor}{RGB}{169, 183, 198}
\definecolor{pageColor}{RGB}{43, 43, 43}

\usepackage{mathtools}

\makeatletter
\let\mytagform@=\tagform@
\def\tagform@#1{\maketag@@@{\color{fontColor}(#1)}}
\makeatother

\renewcommand\theequation{{\color{fontColor}\arabic{equation}}}



\usepackage{graphicx}
\graphicspath{{images/}}
\usepackage{amssymb,amsfonts,amsmath,amsthm}
\usepackage{mathtext}
\usepackage{cite}
\usepackage{enumerate}
\usepackage{float}
\usepackage[pdftex,unicode,colorlinks = true,linkcolor = white]{hyperref}
\usepackage{indentfirst}
\usepackage{placeins}
\bibliographystyle{unsrt}
\usepackage{makecell}
\usepackage{ulem}
\usepackage{longtable}
\usepackage{multirow}
\usepackage{multicol}


\usepackage{fancyhdr}
\pagestyle{fancy}
\fancyhf{}
\fancyhead[R]{\thepage}
\fancyheadoffset{0mm}
\fancyfootoffset{0mm}
\setlength{\headheight}{17pt}
\renewcommand{\headrulewidth}{0pt}
\renewcommand{\footrulewidth}{0pt}
\fancypagestyle{plain}{
\fancyhf{}
\rhead{\thepage}}

\usepackage{geometry}
\geometry{left=1.5cm}
\geometry{right=1.5cm}
\geometry{top=2.4cm}
\geometry{bottom=2.4cm}

\author{Aleksandr Zhuravlyov}
\title{Numerical model of steady state and transient flow}
\date{\today}


\usepackage {titlesec}
\titleformat{\chapter}{\thispagestyle{myheadings}\centering\hyphenpenalty=10000\normalfont\huge\bfseries}{
\thechapter. }{0pt}{\Huge}
\makeatother


\usepackage{nomencl}
\makenomenclature    % Закомментируйте, если перечень не нужен
%"/usr/texbin/makeindex" %.nlo -s nomencl.ist -o %.nls
\renewcommand{\nomname}{Перечень условных обозначений}
\renewcommand{\nompreamble}{\markboth{}{}}
\newcommand*{\nom}[2]{#1~- #2\nomenclature{#1}{#2}}

\setlength{\columnseprule}{0.4pt}
\setlength{\columnsep}{50pt}
\def\columnseprulecolor{\color{fontColor}}


\begin{document}

    \pagecolor{pageColor}
    \color{fontColor}
    %\maketitle
    %\newpage
    %\tableofcontents{\thispagestyle{empty}}
    %\newpage

    \begin{multicols}{2}
        \begin{center}
        {\large \textbf{FVM Steady State}}
        \end{center}
        %
        \begin{eqnarray}
            \label{eq:poisson_integral}
            \oint \limits_{\Omega} \rho \frac{k}{\mu} \vec{\nabla}P d\vec{\Omega} = 0,
        \end{eqnarray}
        %
        \begin{eqnarray}
            \label{eq:poisson_integral_bound}
            P \Big|_\Gamma = P_\Gamma,
        \end{eqnarray}
        %
        \begin{eqnarray}
            \label{eq:poisson_integral_num}
            \beta_{-}\Delta_{P-} - \beta_{+}\Delta_{P+} = 0,
        \end{eqnarray}
        %
        \begin{eqnarray}
            \label{eq:beta}
            \beta = - \frac{\overline{\lambda} \Delta \Omega}{\Delta L},
        \end{eqnarray}
        %
        \begin{eqnarray}
            \label{eq:lambda}
            \lambda = \rho \frac{k}{\mu},
        \end{eqnarray}
        %
        \begin{eqnarray}
            \label{eq:delta_P_num}
            \Delta_{P} = P_{+} - P_{-},
        \end{eqnarray}
        %
        \begin{eqnarray}
            \label{eq:Consumption_poisson_integral}
            G_{out} = \beta_-\Delta_{P-} \Big|_ {N-1}.
        \end{eqnarray}

        \columnbreak
        \begin{center}
        {\large \textbf{FVM Transient flow}}
        \end{center}
        %
        \begin{eqnarray}
            \label{eq:conductivity_integral}
            \int \limits_{V} \left( \phi \rho^\prime + \phi^\prime \rho \right) \frac{\partial P}{\partial t} d V = \oint \limits_{\Omega} \rho \frac{k}{\mu} \vec{\nabla}P d\vec{\Omega},
        \end{eqnarray}
        %
        \begin{eqnarray}
            \label{eq:conductivity_bound}
            P\left(t\right) \Big|_\Gamma = P_\Gamma\left(t\right),
        \end{eqnarray}
        %
        \begin{eqnarray}
            \label{eq:conductivity_num}
            \alpha \Delta^{t}_{P} + \beta^n_{-}\Delta^{n+1}_{P-} - \beta^n_{+}\Delta^{n+1}_{P+}= 0,
        \end{eqnarray}
        %
        \begin{eqnarray}
            \label{eq:delta_P_t_num}
            \Delta^{t}_{P} = P^{n+1} - P^{n},
        \end{eqnarray}
        %
        \begin{eqnarray}
            \label{eq:alpha}
            \alpha = \left( \phi \rho^\prime + \phi^\prime \rho \right) \frac{\Delta V}{\Delta t},
        \end{eqnarray}
        %
        \begin{eqnarray}
            \label{eq:Consumption_conductivity_integral}
            G_{out}^{n+1} = \beta^n_-\Delta^{n+1}_{P-} \Big|_ {N-1}.                        
        \end{eqnarray}
    \end{multicols}

    \vspace{1.cm}

    \begin{multicols}{2}
        \begin{center}
        {\large \textbf{Analytical Steady State}}
        \end{center}

        \begin{eqnarray}
            \label{eq:poisson_diff}
            \frac{d}{dx}\rho\frac{k}{\mu} \frac{dP}{dx} = 0,
        \end{eqnarray}
        %
        \begin{eqnarray}
            \label{eq:poisson_diff_bound}
            P\left(0\right) = P_{in}, \; P\left(L\right) = P_{out},
        \end{eqnarray}
        %
        \begin{eqnarray}
            \label{eq:consumption_poisson_diff_geniral}
            G = - \frac{1}{L} \int \limits_{P_{in}}^{P_{out}} \rho\frac{k}{\mu} dP,
        \end{eqnarray}
        %
        \begin{eqnarray}
            \label{eq:poisson_dencity_particular}
            \rho = a_{\rho}P + b_{\rho},
        \end{eqnarray}
        %
        \begin{eqnarray}
            \label{eq:poisson_permeability_particular}
            k = \theta_{3}P^3 + \theta_{2}P^2 + \theta_{1}P + \theta_{0},
        \end{eqnarray}
        %
        \begin{eqnarray}
            \label{eq:poisson_rho_k_mu__particular}
            \begin{gathered}
                \rho k = a_{\rho}\theta_{3}P^4 + \left( a_{\rho}\theta_{2} + b_{\rho}\theta_{3}\right)P^3 +\\
                + \left(a_{\rho}\theta_{1} + b_{\rho}\theta_{2}\right)P^2 +
                \left(a_{\rho}\theta_{0} + b_{\rho}\theta_{1}\right)P + b_{\rho}\theta_{0},
            \end{gathered}
        \end{eqnarray}
        %
        \begin{eqnarray}
            \label{eq:consumption_poisson_diff_particular}
            \begin{gathered}
                G = \frac{a_{\rho}\theta_{3}}{5L\mu} \left(P_{in}^5 -P_{out}^5 \right) +\\
                +\frac{a_{\rho}\theta_{2} + b_{\rho}\theta_{3}}{4L\mu} \left(P_{in}^4 -P_{out}^4 \right) +\\
                +\frac{a_{\rho}\theta_{1} + b_{\rho}\theta_{2}}{3L\mu} \left(P_{in}^3 -P_{out}^3 \right) +\\
                +\frac{a_{\rho}\theta_{0} + b_{\rho}\theta_{1}}{2L\mu} \left(P_{in}^2 -P_{out}^2 \right) +
                \frac{b_{\rho}\theta_{0}}{L\mu} \left(P_{in} -P_{out} \right).
            \end{gathered}
        \end{eqnarray}

        \columnbreak

        \begin{center}
        {\large \textbf{Inverse Problem}}
        \end{center}
        %
        \begin{eqnarray}
            \label{eq:functional}
            F\left(f\left(P\right)\right) =\sum_{j=1}^M \left(G^j\left(f\left(P\right)\right) - \tilde{G}^j\right)^2,
        \end{eqnarray}
        %
        \begin{eqnarray}
            \label{eq:functional_parameters}
            F\left(\theta\right) =\sum_{j=1}^M \left(G^j\left(\theta\right) - \tilde{G}^j\right)^2,
        \end{eqnarray}
        %
        \begin{eqnarray}
            \label{eq:minimization_functional_parameters}
            f(P) = \arg  \min_{\theta}  F.
        \end{eqnarray}

    \end{multicols}

    \begin{center}
    {\large \textbf{Steady State Analytical Inverse Problem}}
    \end{center}
    %
    \begin{eqnarray}
        \label{eq:consumption_poisson_diff_particular_d}
        G = \sum \limits_{i=0}^{3} \frac{\partial G}{\partial \theta_i}\theta_i, \;\;
        \frac{\partial G}{\partial \theta_{i}} =
        \frac{a_{\rho}}{\left(i+2\right) L\mu} \left(P_{in}^{i+2} -P_{out}^{i+2} \right)+
        \frac{b_{\rho}}{\left(i+1\right)L\mu} \left(P_{in}^{i+1} -P_{out}^{i+1} \right),
    \end{eqnarray}
    %
    \begin{eqnarray}
        \label{eq:least_squares_analytical_1}
        \sum_{j=1}^M \left(\sum_{n=0}^{3} \frac{\partial G^j}{\partial \theta_n}\theta_n - \tilde{G}^j\right)\frac{\partial G^j}{\partial \theta_i} = 0,
    \end{eqnarray}
    %
    \begin{eqnarray}
        \label{eq:least_squares_analytical_2}
        \sum_{j=1}^M \sum_{n=0}^{3} \frac{\partial G^j}{\partial \theta_i} \frac{\partial G^j}{\partial \theta_n}\theta_n =\sum_{j=1}^M \frac{\partial G^j}{\partial \theta_i} \tilde{G}^j.
    \end{eqnarray}


\end{document}
